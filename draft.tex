%\documentclass[10pt,a4paper]{article}

\documentclass[10pt,a4paper]{llncs}

\usepackage{makeidx}  % allows for indexgeneration
\usepackage[utf8]{inputenc}
\usepackage[cmex10]{amsmath}
\usepackage{amsfonts}
\usepackage{hyperref}
\usepackage{color}
\usepackage{cleveref}
\usepackage{graphicx}
\usepackage{caption}
\usepackage{subcaption}

\newcommand{\todo}[1]{{\color{red}\textsf{\textbf{TODO}}: #1}}
\newcommand{\uri}[1]{\texttt{#1}}
\newcommand{\word}[1]{`#1'}
\newcommand{\model}[1]{\textsf{#1}}
%\newcommand{\Cref}[1]{\ref{#1}}
\crefname{table}{Table}{Tables}

\title{Enhancing Named Entity Recognition and Disambiguation via Multilingual Data Fusion}
\author{Renan Dembogurski}
\date{\today}

\begin{document}
\maketitle

\begin{abstract}
\end{abstract}

\section{Introduction}

Temos que decidir se vamos no caminho do Data Fusion ou não. Talvez o caminho seja: Enhancing Named Entity Recognition and Disambiguation via Multilingual Data Fusion. Algo assim. Mas temos que tomar cuidado pra não crescer o problema muito. Acho que nos mantemos no seguinte:

pegamos tipos de outras línguas, avaliamos a acurácia da fusão.
usando tipos extendidos, extraímos um corpus de NER, treinamos e avaliamos.
reportamos que nossa cobertura com tipos extendidos melhorou.
assuma-se portugues como língua sendo avaliada. usando sameAs, pegamos parágrafos de outras línguas (ex: Ingles), pegamos as entidades nos parágrafos em ingles, e pegamos os labels em portugues.
avaliamos a eficacia de desambiguacao com o indice normal do portugues, e com o indice expandido com contexto do ingles.


\section{Related Work}


\section{Methodology}

\section{Experiments}

...


\section{Conclusion}


\bibliographystyle{splncs03}
\bibliography{eswc}

\end{document}
