\documentclass[10pt,a4paper]{article}
%\documentclass[10pt,a4paper]{llncs}

\usepackage{makeidx}  % allows for indexgeneration
\usepackage[utf8]{inputenc}
\usepackage[cmex10]{amsmath}
\usepackage{amsfonts}
\usepackage{hyperref}
\usepackage{color}
\usepackage{cleveref}
\usepackage{graphicx}
\usepackage{caption}
\usepackage{subcaption}
\usepackage{color}

\definecolor{gray}{rgb}{0.5,0.5,0.5}
\newcommand{\todo}[1]{{\color{red}\textsf{\textbf{TODO}}: #1}}
\newcommand{\uri}[1]{\texttt{#1}}
\newcommand{\word}[1]{`#1'}
\newcommand{\model}[1]{\textsf{#1}}
\newcommand{\copypaste}[1]{``{\color{gray}{#1}}''}
%\newcommand{\Cref}[1]{\ref{#1}}
\crefname{table}{Table}{Tables}

\title{Enhancing Named Entity Recognition and Disambiguation via Multilingual Data Fusion}
\author{TBD}
\date{\today}

\begin{document}
\maketitle

\begin{abstract}
\end{abstract}

\section{Introduction}

%===== DICAS ======
%- Digite uma frase por linha, assim fazer merge fica muito mais fácil com ferramentas de diff (tipo meld ou winmerge)
%- Sempre marque copy+paste com aspas e de preferencia com \verbatim{} ou \copypaste

Summary:
\begin{itemize}
\item pegamos tipos de outras línguas, avaliamos a acurácia da fusão. (10\% do valor do artigo)
\item usando tipos extendidos, extraímos um corpus de NER, treinamos e avaliamos.
\item reportamos que nossa cobertura com tipos extendidos melhorou.  (40\% do valor do artigo)
\item assuma-se portugues como língua sendo avaliada. usando sameAs, pegamos parágrafos de outras línguas (ex: Ingles), pegamos as entidades nos parágrafos em ingles, e pegamos os labels em portugues. 
avaliamos a eficacia de desambiguacao com o indice normal do portugues, e com o indice expandido com contexto do ingles. (50\% do valor do artigo)
\end{itemize}


\section{Related Work}

\paragraph{Data Fusion}
Data Fusion is... (quickly paraphrase from paper by Bleiholder and Naumann) - 1 paragraph

Wikipedia has cross-language links... allows us to complement one language with information from another.
Palmero et al. \copypaste{present a resource obtained by automatically mapping templates in 25 languages.} \cite{iknow13palmero}

Other authors have also worked on extending types...
Gangemi et al. present T\'{ipalo}, a tool that \copypaste{identifies the most appropriate types for an entity by interpreting its natural language definition, which is extracted from its corresponding Wikipedia page abstract.} \cite{iswc12gangemi}

%Classifying the Wikipedia Articles into the OpenCyc Taxonomy
%http://ceur-ws.org/Vol-906/paper2.pdf

\paragraph{Cross-language}
Cross-language entity linking is the problem ``where documents and named entities are in a different language than that used for the content of the reference knowledge base.''
Their \copypaste{best model achieves performance which is 94\% of a strong monolingual baseline.}
http://www.aclweb.org/anthology/I/I11/I11-1029.pdf

\todo{get/generate bibtex}
Cross Lingual Entity Linking with Bilingual Topic Model
Tao Zhang, Kang Liu and Jun Zhao
http://ijcai.org/papers13/Papers/IJCAI13-327.pdf

\todo{get/generate bibtex}
Analysis and Refinement of Cross-lingual Entity Linking
http://nlp.cs.rpi.edu/paper/clef2012.pdf

We are rather looking at complementing resource-poor languages. 
We show that our best model actually improves the results.
We do not perform only entity-linking but the full annotation (NER+NEL)

\section{Methodology}

\section{Experiments}

\subsection{Data Fusion Evaluation}

\subsection{NER Evaluation}

\subsection{Cross-language Entity Linking Evaluation}

\section{Conclusion}


\bibliographystyle{splncs03}
\bibliography{eswc}

\end{document}
